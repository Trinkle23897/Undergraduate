\section{性能评测}

\subsection{运行性能测试}
\qquad
作者通过比对最小费用最大流算法\emph{MCMF}(能得出最优解)、\emph{Dinic}网络流算法(能得出可行解)、基于规则的布线算法\emph{Rule}(能得出次优解)的运行时间,来比较这三种方法的运行效率。

%我是把布线之后点与点的宽度估计成了logN
理论上分析,对于n个点,m条边并且容量都为1的\emph{MCMF},时间复杂度为$O(nm^2)$,而\emph{Dinic}的时间复杂度为$O(\min{(n^{2/3},m^{1/2})}\cdot m)$。在这个问题中,记$n^2$为需要连接的点的个数,$n\in [1,80]$,$N$为电路板的边长,结果表明$\frac{N - 1}{n + 1}\in [\frac{n}{4}, \frac{n}{3}]$,因此可以知道$N = \Theta(n^2)$。

因此复杂度可估计为\emph{MCMF}:$O(n^6)$,\emph{Dinic}:$O(n^3)$。

而人工设计的基于规则布线的方法几乎是正比于布线的总长度,因此时间复杂度为$O(n^2)$。

运行效率具体如下表所示:\footnote{运行环境:Intel(R) i5-5200U CPU 2.20GHz}

\begin{table}[H]
\label{tab:1}
\centering
\begin{tabular}{|c|c|c|c|}
\hline
\emph{n} & \emph{MCMF} & \emph{Dinic} & \emph{Rule}\\
\hline
5 & 0.040& 0.040& 0.040\\
\hline
15 & 0.308&0.056 & 0.040\\
\hline
25 & 7.504& 0.404& 0.064\\
\hline
35 &3'25'' & 2.276& 0.148\\
\hline
45 &24'4'' & 14.656& 0.272\\
\hline
55 &数小时 &66.584 & 0.484\\
\hline
65 &数小时 &7'13'' & 0.904\\
\hline
75 &约$8\sim 9h$&约$1h$  & 1.712\\
\hline
\end{tabular}
\caption{性能测试,时间若未注明则以秒计}
\end{table} 

可以看到,在实际运行中,费用流算法极其耗时,数据一旦超过40就很难在短时间内出结果;\emph{Dinic}算法在数据大的时候还是能在数分钟以内就出解的;规则布线方法效率最高,在不到数秒内就能出解。

\subsection{运行效果分析}

\begin{table}[H]
\label{tab:2}
\centering
\begin{tabular}{|c|c|c|c|c|c|}
\hline
\emph{n} & \emph{MCMF} & \emph{Dinic} & \emph{Rule} & $\frac{Rule-MCMF}{MCMF}\%$&$\frac{Dinic-MCMF}{MCMF}\%$\\
\hline
5 & 79 & 79 & 79 & 0.000&0.000 \\
\hline
15 & 3862&3910&3879&0.440&1.243 \\
\hline
25 & 27394	&   27566	 &  27460	&0.241& 0.628\\
\hline
35 &101775	&  102051	&  101920	&	0.142&0.271\\
\hline
45 & 273183	 & 273811	&  273433	&	0.092 &0.230\\
\hline
55 & 602856	 & 603975	&  603241	&0.064 &0.186\\
\hline
65 & 1167116	& 1168932	& 1167662	&	0.047&0.156\\
\hline
75 & 2057345	& 2060192	& 2058082	&	0.036&0.138\\
\hline
\end{tabular}
\caption{效果分析}
\end{table}

\qquad 从表\ref{tab:2}中可以看出,\emph{Dinic}算法所得出的解在数据范围大的时候,误差均小于$1\%$;相比之下,规则布线方法的结果总是要优于\emph{Dinic},并且相对最优解的误差在数据越大的时候越小,在数据大的时候,误差更是小于万分之五。可以说这是一个很优秀的方法。
