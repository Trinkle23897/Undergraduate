\section{使用方法}
\subsection{编译及运行}
	\qquad
	本项目依赖一些第三方库:
	\begin{itemize}
		\item Boost
		\item OpenCV
	\end{itemize}

	并且使用 CMake 来进行 Makefile 的生成,可以在进入 src 目录后运行如下代码进行编译
	\begin{minted}{bash}
    mkdir build
    cd build
    cmake ../
    make
	\end{minted}

	编译完成后使用 \mintinline{bash}{./main} 来运行程序,或可以使用 \mintinline{bash}{./main --help} 来查看帮助。

	如果你想用基于网络流的算法计算节点大小为18x18的布线方案,可以运行如下

	\begin{minted}{bash}
    ./main -n 18 -a flow
	\end{minted}

	如果你想保存路径信息,并且不想显示窗口,则可以运行如下

	\begin{minted}{bash}
    ./main -n 18 -p 18x18.data --no-window
	\end{minted}

	如果你想将结果保存为大小为3000x3000的图片,并在窗口以700x700的大小显示,则可以运行如下

	\begin{minted}{bash}
    ./main -n 18 --img-size 3000 --win-size 700
	\end{minted}

	如果想要查看具体的计算过程,则可以运行如下

	\begin{minted}{bash}
    ./main -n 30 --show-step -d 500
	\end{minted}
\subsection{具体参数}
	\begin{itemize}
	\item[-h] 查看帮助。
	\item[-n] 指定x方向以及y方向有多少个节点,默认为12。
	\item[-a] 指定算法,有基于规则(rule)和基于网络流(flow)两个方法,默认为 rule。
	\item[-p] 将结果存储到指定文件(若没有这个参数则不存储)。
	\item[-i] 从指定文件读入结果并且显示。
	\item[-o] 将结果以图片形式存储到文件。
	\item[-{}-img-size] 指定存储的图片大小,默认为2000px。
	\item[-{}-win-size] 指定显示的窗口大小,默认为800px。
	\item[-{}-no-window] 不在窗口显示结果。
	\item[-{}-show-step] 显示计算的具体步骤。
	\item[-d] 指定两个计算步骤的显示间隔时间(单位为ms,默认为1)。
	\end{itemize}

